\documentclass[12pt]{article}
\usepackage[pdftex,pagebackref,letterpaper=true,colorlinks=true,pdfpagemode=none,urlcolor=blue,linkcolor=blue,citecolor=blue,pdfstartview=FitH]{hyperref}

\usepackage{amsmath,amsfonts}
\usepackage{graphicx}
\usepackage{color}


\setlength{\oddsidemargin}{0pt}
\setlength{\evensidemargin}{0pt}
\setlength{\textwidth}{6.0in}
\setlength{\topmargin}{0in}
\setlength{\textheight}{8.5in}

\setlength{\parindent}{0in}
\setlength{\parskip}{5px}

\input{macrosblog}

\begin{document}
    参考:\href{https://zhuanlan.zhihu.com/p/53304713}{浅谈KAM定理的证明},
     \href{https://zhuanlan.zhihu.com/p/33763445}{关于经典KAM理论}, 
     \href{https://zhuanlan.zhihu.com/p/57840760}{弱KAM理论简介},
     Knauf, A. (2016). H. Scott Dumas: “The KAM Story: A Friendly Introduction to the Content, History, and Significance of Classical Kolmogorov-Arnold-Moser Theory.” Jahresbericht Der Deutschen Mathematiker-Vereinigung. https://doi.org/10.1365/s13291-015-0128-8
     拟微分算子和Nash-Moser定理
     \section{1.简介}
     KAM理论是研究近可积哈密顿系统形态的理论,揭示了太阳系稳定性(一定条件下)等问题的原因,证明这一
     定理的方法与 Nash-Moser iteration 密切相关, 都是改进的牛顿法,
     Nash-Moser iteration 在偏微分方程,动力系统,微分几何中都有重要应用 。同为20世纪数学重大进展,
     因此作者认为将他们一同叙述是有益的,本文将给出KAM定理的简述,并给出 Nash-Moser iteration 的一个
     证明概要, 展示他们都是更大的扰动方法中的一员。
\end{document}
