\documentclass[12pt]{article}
\usepackage[pdftex,pagebackref,letterpaper=true,colorlinks=true,pdfpagemode=none,urlcolor=blue,linkcolor=blue,citecolor=blue,pdfstartview=FitH]{hyperref}

\usepackage{amsmath,amsfonts}
\usepackage{graphicx}
\usepackage{color}


\setlength{\oddsidemargin}{0pt}
\setlength{\evensidemargin}{0pt}
\setlength{\textwidth}{6.0in}
\setlength{\topmargin}{0in}
\setlength{\textheight}{8.5in}

\setlength{\parindent}{0in}
\setlength{\parskip}{5px}

%%%%%%%%% For wordpress conversion

\def\more{}

\newif\ifblog
\newif\iftex
\blogfalse
\textrue


\usepackage{ulem}
\def\em{\it}
\def\emph#1{\textit{#1}}

\def\image#1#2#3{\begin{center}\includegraphics[#1pt]{#3}\end{center}}

\let\hrefnosnap=\href

\newenvironment{btabular}[1]{\begin{tabular} {#1}}{\end{tabular}}

\newenvironment{red}{\color{red}}{}
\newenvironment{green}{\color{green}}{}
\newenvironment{blue}{\color{blue}}{}

%%%%%%%%% Theorems and proofs

\newtheorem{exercise}{Exercise}
\newtheorem{theorem}{Theorem}
\newtheorem{lemma}[theorem]{Lemma}
\newtheorem{definition}[theorem]{Definition}
\newtheorem{corollary}[theorem]{Corollary}
\newtheorem{proposition}[theorem]{Proposition}
\newtheorem{example}{Example}
\newtheorem{remark}[theorem]{Remark}
\newenvironment{proof}{\noindent {\sc Proof:}}{$\Box$ \medskip} 


\begin{document}
    虽然傅里叶早在18世纪就提出了将函数用三角级数表示以求解热方程的想法,但这一方法的有效性一直悬而未决。
虽然有部分结果,但条件太过严格。事实上,直到Carleson的定理横空出世前,人们对此表示怀疑并且连
连续函数的情况都不明了。
    先给出Carleson-Hunt定理的一个版本:
    设U是任何包含原点的有界开集, 则对于$f\in L^2, 当 k\rightarrow\infty时,(\xi_{2kU}f\hat)\check$几乎处处收敛至f. 如果U是内含原点的方体, 则上述结论可以推广到$f\in L^p,1<p<\infty$。

\section{当我们在谈论收敛时是指什么?}
    微积分中级数的收敛就是部分和有极限,对于$f\in L^1$,我们定义其傅里叶系数及级数为:
    \begin{eqnarray*}
\hat{f}(k)=\int_{0}^{1} f(x) e^{-2 \pi i k x} d x
\quad \sum_{k=-\infty}^{\infty} \hat{f}(k) e^{2 \pi i k x}
    \end{eqnarray*}
    然后我们就可以定义傅里叶级数的部分和:
    \begin{equation}
S_{N} f(x)=\sum_{k=-N}^{N} \hat{f}(k) e^{2 \pi i k x}
\end{equation}
    基于实分析的技术,我们考虑两种收敛:
    \begin{equation}
\begin{array}{l}{\text { (1) Does } \lim _{N \rightarrow \infty}\left\|S_{N} f-f\right\|_{p}=0 \text { for } f \in L^{p}(\mathbb{T}) ?} \\ {\text { (2) Does } \lim _{N \rightarrow \infty} S_{N} f(x)=f(x) \text { almost everywhere if } f \in L^{p}(\mathbb{T}) ?}\end{array}
\end{equation}
前者称为范数收敛,后者为逐点收敛

\section{历史的玩笑}
    1915年,Lusin 猜想对任意函数$f\in L^2[-\Pi,\Pi]$,其傅里叶级数几乎处处收敛到原函数。
    1926年,Kolmogorov 证明存在函数$f\in L^1[-\Pi,\Pi]$,傅里叶级数处处发散。
    1927年,Riesz 证明对任意函数$f\in L^p[-\Pi,\Pi](1<p<\infty)$,傅里叶级数$L^p$范收敛到原函数。
    (这个是Hilbert变换的一个推论)
    1966年,Carleson 证明了 Lusin 猜想。同年, Kathane 和 katznelson 证明存在在每个零测集上傅里叶级
数发散的函数。
    1968年,Hunt 把 Carleson的定理推广到$L^p(1<p<\infty)$。
    人们兴高采烈以为高维也唾手可得时反例就来了。
    1971年,Fefferman 在两篇文章中利用Kakeya set\em{Kakeya set是$\R^n$中一个包含所有
    方向长度为1线段的集合,它的测度下限为0,它的维度至今仍是公开问题}对\eqref{eq:1}给出了反例,证明了\eqref{eq:2},
    \eqref{eq:3}依然公开,即使是现在考虑的比\eqref{eq:3}更光滑的Boncher-Riesz也搞不定。
\begin{equation}\label{eq:1}f(x, y)=\lim _{M, N \rightarrow \infty} \sum_{|m| \leq M_{i}|n| \leq N} a_{m n} e^{i(m x+n y)}\end{equation}
\begin{equation}\label{eq:2}f(x, y)=\lim _{M \rightarrow \infty} \sum_{|m| .} \sum_{|n| \leq M} a_{m n} e^{i(m x+n y)}\end{equation}
\begin{equation}\label{eq:3}f(x, y)=\lim _{R \rightarrow \infty} \sum_{m^{2}+n^{2} \leq R} a_{m n} e^{i(m x+n y)}\end{equation}
\section{充分与必要}
\begin{lemma}
    $S_{N}f(x)$范数收敛和逐点收敛到$f(x)$的充要条件分别为
    $$
\left\|S_{N} f\right\|_{p} \leq C_{p}\|f\|_{p}
$$
$$
\left\|\sup_N|S_{N} f|\right\|_{p} \leq C_{p}\|f\|_{p}
$$
\end{lemma}
必要性来自以下定理:
\begin[一致有界定理]{lemma}
    令$x$为 Banach 空间,$Y$为 赋范向量空间且$(T_i)_{i\in I}为一族$x$到$Y$的连续线性算子,假设
    \[\sup_{i\in I}\|T_i(x)<\infty\ \forall x\in X\]
    则
    \[\sup_{i\in I}\|T_i\|<\infty]
    即$T_i$为有界算子
\end{lemma}
注意$\|T_i\|=\sup_{\|x\|_X\leq 1\\x\in X}$
我们来处理充分性,为简便计,只说明逐点收敛情况,
定义
\[L_f:=\limsup_N\rightarrow\infty\|f(x)-S_N f(x)\|\rightarrow 0\ a.e.\]
傅里叶级数对施瓦茨函数成立,我们取施瓦茨函数$g$使得$\|f-g\|\leq\epsilon$
有:\[L_f\leq\|f-g\|+\mathcal{C}(f-g)\]
    \end{document}
